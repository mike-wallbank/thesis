% FD analysis chapter -- using shower reconstruction to distinguish pi0s and electrons?
% Target: 10 pages

%----------------------------------------------------------------------------------------------------------------------------------------------------------------------------
%\chapter{Electron Reconstruction for $\nu_e$ Oscillation Signal at the DUNE Far Detector}\label{chap:FDAnalysis}
\chapter{The $\nu_e$ Oscillation Signal at the DUNE Far Detector}\label{chap:FDAnalysis}

A primary aim of the DUNE experiment, as discussed in Chapter~\ref{chap:DUNE}, is to make precision measurements of the PMNS matrix parameters describing neutrino oscillations by searching for electron neutrino appearance from the predominantly muon neutrino beam (i.e. $\nu_{\mu} \rightarrow \nu_e$ oscillation, described by Equation~\ref{eq:ElectronNeutrinoAppearance}).  This channel is of critical importance for all oscillation-related physics and thus requires very efficient reconstruction and selection.  Methods developed to provide reconstruction of these events were discussed in detail in Chapter~\ref{chap:LArTPCReconstruction} and utilised in this present chapter in the selection of simulated charged-current (CC) $\nu_e$ events ($\nu_e$CC) in the DUNE far detector.

The selection presented in the following sections represents the very first generation analysis for the DUNE experiment and serves primarily to demonstrate the principle of selecting these events in a large LArTPC.  Much work is needed to advance the analysis to comply with DUNE requirements and many improvements may be expected as further developments progress.  It should also be noted the reconstruction discussed in Chapter~\ref{chap:LArTPCReconstruction} is not the only solution and various other techiniques, primarily using the Pandora toolkit \cite{Pandora2015}, have been assessed, notably in Section~\ref{sec:FDCut}.  The selection presented in Section~\ref{sec:FDMVA} does utilise the novel reconstruction detailed in this thesis however, due to significant recent progress, it is likely the selection will continue to explore all reconstructions in LArSoft and take advantage of the continuing developments.  This outlook will be briefly discussed in Section~\ref{sec:FDOutlook}.

%----------------------------------------------------------------------------------------------------------------------------------------------------------------------------
\section{Cut-Based Tuning}\label{sec:FDCut}

This section details the early developments of a selection which utilises only Pandora reconstruction and a cut-based approach.  This procedure shows promise at performing as well as the more established multi-variate analysis (MVA) method, discussed in Section~\ref{sec:FDMVA}, but is used here mainly to tune the selection.

%----------------------------------------------------------------------------------------------------------------------------------------------------------------------------
\subsection{Selection}\label{sec:FDCutSelection}

%----------------------------------------------------------------------------------------------------------------------------------------------------------------------------
\subsection{Fiducial Volume Tuning}\label{sec:FDCutFV}

%----------------------------------------------------------------------------------------------------------------------------------------------------------------------------
\section{MVA-Based Selection}\label{sec:FDMVA}

%----------------------------------------------------------------------------------------------------------------------------------------------------------------------------
\subsection{MVA Input Variables}\label{sec:FDMVAVariables}

%----------------------------------------------------------------------------------------------------------------------------------------------------------------------------
\subsection{Analysis Performance}\label{sec:FDMVAPerformance}

%----------------------------------------------------------------------------------------------------------------------------------------------------------------------------
\section{Outlook for Future Selections}\label{sec:FDOutlook}
