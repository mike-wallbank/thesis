% ************************** Thesis Abstract *****************************
% Use `abstract' as an option in the document class to print only the titlepage and the abstract.
\begin{abstract}

  Neutrino physics is approaching the precision-era, with current and future experiments aiming to perform highly accurate measurements of the parameters which govern the phenomenon of neutrino oscillations.  The ultimate ambition with these results is to search for evidence of CP-violation in the lepton sector, currently hinted at in the world-leading analyses from current experiments, which may explain the dominance of matter over antimatter in the Universe.

  The Deep Underground Neutrino Experiment (DUNE) is a future long-baseline experiment based at Fermi National Accelerator Laboratory (FNAL), with a far detector at the Sanford Underground Research Facility (SURF) and a baseline of 1300~km.  In order to make the required precision measurements, the far detector will utilise a detector consisting of 40~kton liquid argon and an embedded time projection chamber.  This promising technology is still in development and, since each detector module is around a factor 15 larger than any previous experiment employing this design, prototyping the detector and design choices is critical to the success of the experiment.  The 35-ton experiment was constructed for this purpose and will be described in detail in this thesis.  The outcomes of the 35-ton prototype are already influencing DUNE and, following the successes and lessons learned from the experiment, confidence can be taken forward to the next stage of the DUNE programme.

  The main oscillation signal at DUNE will be electron neutrino appearance from the muon neutrino beam.  High-precision studies of these $\nu_e$ interactions requires advanced processing and event reconstruction techniques, particularly in the handling of showering particles such as electrons and photons.  Novel methods developed for the purposes of shower reconstruction in liquid argon are presented with an aim to successfully develop a selection to use in a $\nu_e$ charged-current analysis, and a first-generation selection using the new techniques is presented.
  
\end{abstract}
