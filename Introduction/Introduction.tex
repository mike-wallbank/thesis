% Introductions
% Target lenth: ~2/3 pages

\graphicspath{{Introduction/Figs/}}

%----------------------------------------------------------------------------------------------------------------------------------------------------------------------------
\chapter{Introduction}\label{chap:Introduction}

The theory of elementary particles, the Standard Model of Particle Physics, is an incredibly successful theory which has stood up to every experimental test since it was first formulated in the 1970s \cite{Glashow1961,Weinberg1967}.  The recent discovery of the Higgs boson at CERN \cite{Aad2012,Chatrchyan2012} was the final missing piece and establishes the Standard Model as \textit{the} theory of physical phenomena at the electroweak scale (up to a few hundred GeV) \cite{Shears2012,Bilenky2015}.

There are however many shortcomings to the theory and further theoretical and experimental work is necessary to advance our understanding of fundamental physics \cite{Ellis2012}.  For example, it ignores gravity and requires a quantised theory of gravity to reconcile it with General Relativity.  The observation of `Dark Matter' and `Dark Energy' in Astrophysics and Cosmology cannot be explained using the known particles in the Standard Model and needs an extension of the theory.  It also offers no convincing explanation of the observed domination of matter over antimatter evident in the Universe, given they were created equally in the Big Bang.  Additionally, there are many unresolved theoretical problems within the Standard Model, evidence of a more fundamental underlying theory which may replace it.  It is for this reason that experiments are hoping to find phenomena which may only be understood `Beyond the Standard Model'.

Neutrinos offer the most promising possibilities of new physics and are currently the subject of a great amount of research \cite{Bilenky2015}.  The observation of neutrino oscillations \cite{SuperKamiokande1998,SNO2002}, along with the associated implication of neutrino mass, represents physics which was not included in, or predicted by, the Standard Model.  In recent years the field of neutrino physics has advanced rapidly and there is currently good understanding of most experimental results.  Open questions remain, such as the origin and nature of neutrino mass, the characteristics of neutrino interactions and the exact features of neutrino mixing, and will define the future of the field for many years to come.  This will be discussed in more detail in Chapter~\ref{chap:NeutrinoPhysics}.  There is also the possibility neutrinos may explain the aforementioned matter-antimatter asymmetry through CP-violation in the lepton sector and may even provide a potential dark matter candidate in the possible sterile neutrino.

Future understanding and discoveries in neutrino physics requires precise measurements from highly sensitive experiments.  The future Deep Underground Neutrino Experiment (DUNE) is such an experiment and will be able to contribute towards many of the unanswered questions in the field.  The DUNE experiment, along with its sensitivities to unexplained phenomena, is the subject of Chapter~\ref{chap:DUNE}.  It will use large quantities of liquid argon in order to make the necessary precision measurements and will be the largest experiment using this technology ever built by an order of magnitude.  In order to ensure the experiment is successful and reaches its physics potential, prototyping the technology and detector design is essential.  The experiences of operating such a prototype, the 35~ton experiment, is discussed in Chapter~\ref{chap:35ton}, and additionally in Chapter~\ref{chap:OnlineMonitoring}.

A major challenge in the design choice of DUNE is the successful and detailed reconstruction of particle interactions necessary to make the required measurements.  This is discussed in depth in Chapter~\ref{chap:LArTPCReconstruction}, with emphasis placed on the difficult task of reconstructing showering particles.  The performance of the reconstruction in the selection of the main signal events for DUNE, and an analysis, at this early stage, of the current status of the DUNE software at meeting its required physics goals, is presented in Chapter~\ref{chap:FDAnalysis}.

